\documentclass{article}
\usepackage{graphicx}
\usepackage{cite}
\usepackage{url}
\usepackage{amsmath}
\usepackage{geometry}
 \geometry{
 a4paper,
 left=20mm,
 right=20mm,
 top=20mm,
 bottom=20mm,
 }
 \textwidth 6in
 \textheight 9in
\begin{document}
 {\Large Arithmetic Progression(AP)}
 \vspace{2mm}
 \\
 {\emph Definition:} \hspace{2mm}A sequence in which the difference between any two consecutive term is constant is called as {\emph Arithimetic Progression(AP)}.
 The constant value that defines the difference between any two consecutive terms is called the {\emph Common Difference}.
 \\ Let T be an arithmetic progression that has a as the first term and common difference of d, then we can write it's $n^{th}$ term as:
 \\\vspace{1mm}$T_{n}$ = a + (n-1)d
\vspace{4mm} \\ 
 {\large Sum of Arithimetic Series}
 \vspace{2mm}
\\Let the first term be a and and common difference d, then the sum of arithmetic series for n terms,$S_{n}$ is:
\\\vspace{1mm}$S_{n}$ = a + (a+d) + (a+2d) +...+ (a+(n-2)d) + (a+(n-1)d) 
\\Reversing the series:
\\\vspace{1mm}$S_{n}$ = (a+(n-1)d) + (a+(n-2)d) +...+ (a+2d) + (a+d) + a
\\Adding the corresponding terms, noting that they add up to 2a+(n-1)d everytime:
\\\vspace{1mm}2*$S_{n}$=$\underbrace{(2a+(n-1)d) + ... + (2a+(n-1)d)}_{n terms}$
\\There are n of these terms, so:
\\ \vspace{2mm} 2*$S_{n}$=n*(2a + (n-1)d)
\\ \vspace{2mm} $S_{n}$=$\frac{n(2a + (n-1)d)}{2}$
\\Also, the first term in the series is a, and the last one is a+(n-1)d,
so we can say the sum of the series is the first term plus the last term multiplied by the number of terms divided by 2.
\\\vspace{1mm}i.e, $S_{n}$ =$\dfrac{n}{2}$($a_{1}$+$a_{n}$)
\vspace{5mm}
\\
{\Large Geometric Progression(GP)}
 \vspace{2mm}
 \\
 {\emph Definition:} \hspace{2mm} A sequence in which the ratio of the a term to it's preceding term is the same for any term is called as {\emph Geometric Progression}. 
 The constant value of the ratio is called as the common ratio of the geometric progression .
 \\Let T be an geometric progression that has a as the first term and common ratio is r, then we can write it's $n^{th}$ term as:
 \\$T_{n}$ = a $r^{n-1}$
 \\
 \\
 {\large Sum of Geometric Series } ~\cite{wiki}
 \vspace{2mm}
\\Let $S_n =\displaystyle\sum_{j \mathop = 0}^{n - 1}$ a $r^j$.
\\Then:
\\$(r - 1) S_n =r S_n - S_n$\vspace{1mm}
\\ \hspace*{14mm} = a($r \displaystyle\sum_{j \mathop = 0}^{n - 1} r^j - \displaystyle\sum_{j \mathop = 0}^{n - 1} r^j$)\vspace{1mm}
\\ \hspace*{14mm}  =a($ \displaystyle\sum_{j \mathop = 1}^n r^j - \displaystyle\sum_{j \mathop = 0}^{n - 1} r^j $\vspace{1mm})
\\ \hspace*{14mm}  =a$ \Bigg(r^n + \displaystyle\sum_{j \mathop = 1}^{n-1} r^j - \left({r^0 + \displaystyle\sum_{j \mathop = 1}^{n - 1} r^j}\right) \Bigg) $ \vspace{1mm}
\\ \hspace*{14mm}  =a($r^n - r^0$) \vspace{1mm}
\\ \hspace*{14mm}  =a($r^n - 1 $)\vspace{1mm}
\\
\\ Hence, $S_n$ = a ($\dfrac{r^n -1}{r-1}$)
\vspace{5mm}
\bibliography{citation}{}
\bibliographystyle{plain}
\end{document}
