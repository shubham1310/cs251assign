\documentclass{article}
\usepackage{graphicx}
\usepackage{geometry}
 \geometry{
 a4paper,
 left=20mm,
 right=20mm,
 top=20mm,
 bottom=20mm,
 }
 \textwidth 6in
 \textheight 9in
\begin{document}
 {\Large Arithmetic Progression(AP)}
 \vspace{2mm}
 \\
 {\emph Definition:} \hspace{2mm}A sequence in which the difference between any two consecutive term is constant is called as {\emph Arithimetic Progression(AP)}.
 The constant value that defines the difference between any two consecutive terms is called the {\emph Common Difference}.
 \\ Let T be an arithmetic progression that has a as the first term and common difference of d, then we can write it's $n^{th}$ term as:
 \\$T_{n}$ = a + (n-1)d
\end{document}
